\begin{table*}[t]
\centering
\caption{Sources of uncertainty in $\epsilon$. We calculate $\epsilon$ for the period 2015 to 2100. For each drift-correction method and model, \emph{drift uncertainty} corresponds to the 2nd--98th inter-percentile range: (i) for each projection scenario, calculate the 2nd--98th inter-percentile range of the drift-corrected data, then (ii) calculate the mean of this inter-percentile range by averaging across the projection scenarios. For each projection scenario, \emph{model uncertainty} corresponds to the inter-model range: (i) for each model, calculate the mean of the agnostic-method drift-corrected data, then (ii) calculate the inter-model range. For each model, \emph{scenario uncertainty} corresponds to the inter-scenario range: (i) for each projection scenario, calculate the mean of the agnostic-method drift-corrected data, then (ii) calculate the inter-scenario range. The final three rows contain summary statistics: the minimum, median, and maximum of each column.}
\begin{tabular}{c|rr|rr}
\toprule
\multicolumn{5}{c}{Sources of uncertainty in $\epsilon$ (mm YJ$^{-1}$)} \\ 
\midrule
Model or scenario & \multicolumn{2}{c|}{Drift uncertainty} & \multicolumn{2}{c}{Other uncertainty} \\
 & Linear & Agnostic & Model & Scenario \\
\midrule
ACCESS-CM2 & 1 & 5 &  & 6 \\
ACCESS-ESM1-5 & 0 & 2 &  & 7 \\
CMCC-CM2-SR5 & 3 & 7 &  & 7 \\
CMCC-ESM2 & 3 & 11 &  & 9 \\
CNRM-CM6-1 & 2 & 13 &  & 8 \\
CNRM-ESM2-1 & 2 & 22 &  & 7 \\
CanESM5 & 0 & 2 &  & 10 \\
EC-Earth3-Veg-LR & 2 & 16 &  & 9 \\
IPSL-CM6A-LR & 1 & 3 &  & 8 \\
MIROC6 & 0 & 7 &  & 4 \\
MPI-ESM1-2-HR & 1 & 12 &  & 6 \\
MPI-ESM1-2-LR & 0 & 2 &  & 6 \\
MRI-ESM2-0 & 1 & 2 &  & 6 \\
NorESM2-LM & 1 & 10 &  & 7 \\
NorESM2-MM & 0 & 9 &  & 10 \\
UKESM1-0-LL & 1 & 1 &  & 9 \\
SSP1-2.6 &  &  & 13 &  \\
SSP2-4.5 &  &  & 11 &  \\
SSP3-7.0 &  &  & 12 &  \\
SSP5-8.5 &  &  & 13 &  \\
\midrule
Min & 0 & 1 & 11 & 4 \\
Median & 1 & 7 & 12 & 7 \\
Max & 3 & 22 & 13 & 10 \\
\bottomrule
\end{tabular}
\end{table*}
